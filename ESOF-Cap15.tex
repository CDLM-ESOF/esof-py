\documentclass[12pt, a4paper]{article}

\usepackage[portuguese]{babel}
\usepackage[utf8]{inputenc}
\usepackage{indentfirst}

\title{Capítulo 15 - Monitorando tempo, agendando tarefas e iniciando programas}
\author{Marcela Coury}

\begin{document}
	\maketitle
\begin{abstract}
	Nesse capítulo foi estudado os diferentes módulos do tempo com algumas funções, como usar ferramentas computacionais para ler várias linhas de código ao mesmo tempo e como abrir alguns programas a partir do Python.
\end{abstract}
\section{Bibliotecas relacionadas}
	Biblioteca time:\\time.time() - tempo em segundos a cada inicialização.\\time.sleep() - pausa o programa pelo tempo em segundos da variável indicada.\\round(variável, casas decimais que deseja no número)
	
	Biblioteca datetime:\\datetime.datetime(ano,mês,dia,hora,minutos,segundos)\\strftime() - transforma o tempo de segundos em string.
	
	Biblioteca deltatime
	Biblioteca subprocess: funções para abrir e fechar programas:\\
	subprocess.Popen(caminho) \#Abre o programa\\
	 webbrowser.open() \#Abre uma página da web\\
	
\section{Exemplos utilizando as funções}
	Exemplo 1:\\import datetime\\import time\\datetime.datetime.now()\\dt = datetime.datetime(2017, 5, 2, 21, 35, 0)\\print(dt.year, dt.month, dt.day)
	print(dt.hour, dt.minute, dt.second)\\print(datetime.datetime.fromtimestamp(1000000000))  -a variável em segundos depois de ‘Unix epoch’.\\print(datetime.datetime.fromtimestamp(round(time.time(),0)))
	
	
	Exemplo 2:\\delta = datetime.timedelta(days=11, hours=10, minutes=9, seconds=8)\\delta.days, delta.seconds, delta.microseconds
	(11, 36548, 0)\\delta.total\_seconds()    -total de segundos\\'986948.0'\\str(delta)                -coloca em um formato melhor\\'11 days, 10:09:08'\\
	
	
	Exemplo 3:\\oct21st = datetime.datetime(2015, 10, 21, 16, 29, 0)\\oct21st.strftime('\%Y/\%m/\%d \%H:\%M:\%S')\\'2015/10/21 16:29:00'\\oct21st.strftime('\%I:\%M \%p')\\'04:29 PM'\\
	
	Exemplo 4:\\fileObj = open('hello.txt', 'w')\\fileObj.write('Hello world!')
	12\\fileObj.close()\\import subprocess\\subprocess.Popen(['start', 'hello.txt'], shell=True)\\
	
\section{Parâmetros para a utilização de strftime}
Parâmetro/Significado\\
\%Y	Year with century, as in '2014'\\
\%y	Year without century, '00' to '99' (1970 to 2069)\\
\%m	Month as a decimal number, '01' to '12'\\
\%B	Full month name, as in 'November'\\
\%b	Abbreviated month name, as in 'Nov'\\
\%d	Day of the month, '01' to '31'\\
\%j	Day of the year, '001' to '366'\\
\%w	Day of the week, '0' (Sunday) to '6' (Saturday)\\
\%A	Full weekday name, as in 'Monday'\\
\%a	Abbreviated weekday name, as in 'Mon'\\
\%H	Hour (24-hour clock), '00' to '23'\\
\%I	Hour (12-hour clock), '01' to '12'\\
\%M	Minute, '00' to '59'\\
\%S	Second, '00' to '59'\\
\%p	'AM' or 'PM'\\
\%\%	Literal '\%' character

\section{Projetos}
\subsection{StopWatchData}
\#! python3\\
\# stopwatch.py - A simple stopwatch program.\\
import time\\
print('Pressione ENTER para comecar. Depois, pressione ENTER para iniciar o stopwatch. Pressione Ctrl-C para sair.')\\
input() \# pressione ENTER para comecar\\
print('Started.')\\
startTime = time.time() \# get the first lap's start time\\
lastTime = startTime   \#tempo entre uma rodada e outra\\
lapNum = 1\\
try:\\
while True:\\
input()\\
lapTime = round(time.time() - lastTime, 2)\\
totalTime = round(time.time() - startTime, 2)\\
print('Volta: \#\%s: \%s (\%s)' \%(lapNum, totalTime, lapTime), end='')\\
lapNum += 1\\
lastTime = time.time()  \#reseta a útima volta\\
except KeyboardInterrupt:\\
\#Aperte Ctrl-C\\
print('Done.')\\

\subsection{Prettified Stopwatch}
\#! python3\\
import time \\
import datetime\\
dataFile = open('stopWatchData.txt','a') \#abre o arquivo\\
dt = datetime.datetime.now()\\
timeStamp = dt.strftime('\%d/\%m/\%Y  \%H:\%M')\\
dataFile.write(timeStamp)\\
print('Pressoine ENTER para começar. Depois, pressione ENTER para o stopwatch. Pressione Ctrl-C para parar.')\\
input()\\
print('Comecou...')\\
startTime = time.time()    \# guarda a primeira volta\\
lastTime = startTime\\
lapNum = 1\\
try:\\
while True:\\
input()\\
lapTime = round(time.time() - lastTime, 2)\\
totalTime = round(time.time() - startTime, 2)\\
print('Lap \#\%s: \%s (\%s)' \% (lapNum, totalTime, lapTime), end='')\\
lapNum += 1\\
lastTime = time.time() \# reseta a útima lap time\\
except KeyboardInterrupt:\\
\# Aperte Ctrl-C para parar\\
print('Done.')\\
dataFile.close() \#Fecha o arquivo\\

	
	
	
\end{document}